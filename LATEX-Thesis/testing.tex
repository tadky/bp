\chapter{Testování}  


\section{Testovací scénáře}
Nevalidní token - 
Nevalidní JSON -
Špatný formát archivu -
Nevalidní ID -  

\section{REST rozhraní}
K tomuto testování byla použita knihovna Rest Assured, která nabízí jednoduché prostředí pro vytváření HTTP požadavků a kontrolování správnosti odpovědí. Testy byly vytvořeny pro všechny zdroje, ovšem testují jenom okrajové negativní případy, při kterých nejsou zapotřebí soubory. 

\begin{lstlisting}[caption=Požadavek se špatným tokenem]
@Test
void invalidToken(){
 String badToken = "asdfasdf";
 RestAssured.given().queryParam("token",badToken)
 .contentType(ContentType.JSON).body("test")
 .then().expect().statusCode(401).when().post(context);
}
\end{lstlisting} 

Ty ostatní byly otestovány ručně, jednotlivé testované případy jsou vypsány níže:
\begin{itemize}
	\item Soubor je příliš velký
	\item Kompletní průchod aplikací
	\item Název archivu neodpovídá názvu hlavního texu
	\item Achiv je prázdný
\end{itemize}

Všechny testy prošli pozitivně, tudíž se dá prohlásit, že aplikace reaguje správně na rúzné typy požadavků. 

\section{} 
