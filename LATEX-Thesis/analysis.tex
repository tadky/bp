 \chapter{Analýza}
 
 
 \section{Požadavky} \label{requirements}
 Nyní představíme požadavky kladené na aplikaci. Ty jsou buď požadované zadávajícím nebo odvozené z potřeb, ke kterým bude služba používána, ale i z omezení plynoucích z prostředí, v jakém bude provozována, v tomto případě z prostředí fakulty státní vysoké školy, jmenovitě Fakulty elektrotechnické, ČVUT. \\
 Za pomoci metody FURPS+ rozdělíme požadavky a omezení na funkční a nefunkční. 
 
 \subsection{Funkční požadavky} \label{functional}
 \begin{table}[H]
 	\begin{center}
 		\begin{tabular}{ p{4cm} P{0,7cm} P{7,9cm} }
 			\textbf{Typ} & \multicolumn{2}{l}{\textbf{Požadavky}} \\
 			\midrule[0,15em]
 			Funkčnost 
 			&\multicolumn{2}{l}{\tabitem Webová služba s REST rozhraním pro komunikaci} \\
 			&\multicolumn{2}{l}{\tabitem Umět přijmout požadavky a soubory}\\
 			&\multicolumn{2}{P{7,6cm}}{\tabitem Zkompilování \LaTeX\ souborů s požadovaným nastavením do PDF}\\
 			&\multicolumn{2}{l}{\tabitem Uložení PDF a jeho poskytnutí}\\	
 			&\multicolumn{2}{l}{\tabitem Uživatel může zvolit tyto nastavení:}\\
 			&&\tabitem {\small Přidat vodoznak na každou stranu PDF}\\
 			&&\tabitem {\small Nakonec souboru přidat stránku s předem stanoveným obsahem}\\
 			&&\tabitem {\small Výsledný soubor PDF bude chráněný}\\
 			&&\tabitem {\small Výsledný soubor PDF nebude kopírovatelný nebo tisknutelný}\\
 		\end{tabular}
 	\end{center}
 	\caption{Funkční požadavky}
 	\label{tab:errors}
 \end{table} 
 Služba ma jediný a specifický účel, z toho plyne menší množství funkcionalit. 
 
 \subsection{Nefunkční požadavky}
 
 \begin{table}[H]
 	\begin{center}
 		\begin{tabular}{ p{4cm} P{8,6cm} }
 			\textbf{Typ} & \textbf{Požadavky} \\
 			\midrule[0,15em]
 			Použitelnost & \tabitem Dokumentace k rozhraní na swagger \footnotemark \\
 						& \tabitem Přístup skrz REST Api, pouze s tokenem\footnote{Token je alfanumerický kód, který slouží k ověření identity}\\
 			\midrule		
 			Spolehlivost & \tabitem Služba není kritická\\
 						& \tabitem Uptime 95\% času\\ 
 						& \tabitem Ochrana proti špatnému vstupu\\
 						& \tabitem Autorizace pomocí tokenů\\
 			\midrule
 			Výkon & \tabitem Zvládnutí obsloužení desítek požadavků najednou\\
 						& \tabitem Ukládání výsledných dokumentů po dobu jednoho měsíce\\
 						& \tabitem Maximální doba tvorby dokumentu 5 minut\\	
			\midrule
			Podporovatelnost & \tabitem Služba může být rozšířena o kompilování samotného TeXu a může podporovat vytváření i jiných formátů z \LaTeX u\\	
			\midrule
			Implementace & \tabitem Platforma - Java EE\\
						& \tabitem Komunikace - REST Api\\
						& \tabitem Operační systém - Ubuntu nebo CentOS\\
			\midrule
 			Rozhraní & \tabitem Komunikuje s Moodle a CourseWare, tyto portály posílají data\\
 			\midrule
 			Fyzické & \tabitem Musí být provozováno na serverech ČVUT\\
 	\end{tabular}
 	\end{center}
 	\caption{Nefunkční požadavky}
 	\label{tab:errors}
 \end{table}
\footnotetext{https:/swagger.io/}
 
 Dúvody některých požadavků nemusejí být úplně jasné, proto je zmíníme.
 \begin{itemize}
 	\item Platforma - Je požadováno prostředím vývoje, kde Java EE je hlavní platforma. Tudíž je zajištěna podpora. 
 	\item Operační systém - Také vyžadováno z pohledu podpory, zmíněné systémy jsou na serverech, kde bude služba nasazena nejčastěji používány a tudíž správci tyto systémy znají.   
 	\item Služba není kritická - Poskytuje funkčnost, která nijak neovlivňuje základní funkcionality určených portálů. 
 	\item Musí být provozováno na serverech ČVUT - Jelikož aplikace bude pracovat s dokumenty a popřípadě i informacemi, spojenými s pracovníky školy, je potřeba tato data chránit a uchovávat na vlastních serverech z důvodu GDPR.
 \end{itemize}
 
 Z tabulky je vidět, že služba není nijak kritická a působí jenom jako doplněk výše zmíněných portálů. Musí ovšem splňovat vyšší bezpečnostní nároky souvisejícími prostředím, v jakém se bude používat. 

\section{Existující řešení}
Na základě stanovených požadavků v minulé kapitole přejdeme k analýze již existujících řešení. Momentálně uživatelé Moodle mohou vkládat do svých souborů řádkové příkazy, což není úplně dostačující a nesplňuje to požadavky. Samozřejmě se dají používat pro vytváření PDF z \LaTeX\ souborů kompilátory, které lze stáhnout a používat lokálně. To ovšem není předmětem této práce, a proto se podíváme na webové služby, které více odpovídají potřebám a požadavkům. Pár vybraných si představíme.  

\subsection{OverLeaf}
Placená služba\footnote{https://www.overleaf.com} pro tvorbu \LaTeX\ dokumentů, kterou lze s některými omezeními používat i zdarma. Je velmi oblíbená hlavně kvůli přívětivému prostředí pro tvorbu dokumentů a jejich správu. Také nabízí výhody pro určité zájmové skupiny, nejzajímavější vzhledem k tématu této práce je předplatitelská služba OverLeaf Commons \footnote{https://www.overleaf.com/for/universities}. Ta poskytuje sdílené prostředí se všemi výhodami pro zaměstnance a studenty univerzity. Ale jako všechny ostatní služby je i tato placená, ovšem není jisté, jestli je poskytována všem univerzitám a ani za jakých podmínek.

\subsection{BlueLaTeX}
Open-source služba\footnote{http://www.bluelatex.org/} pro kompilaci LateX souborů. Kompilovat lze na jejich serveru nebo nabízejí kód pro spuštění na vlastním. Kromě samotné serverové implementace je k dispozici i webový klient. Vše je zatím pouze v Beta verzi a samotní tvůrci varují před možnými nedostatky a problémy. Na jejich oficiálních stránkách je poslední aktivita z roku 2015 a zdá se, že tento projekt už není aktivní. Hlavním lákadlem je souběžná spolupráce více lidí na jednom dokumentu a také možnost provozovat server lokálně. 

\subsection{ScienceSoft}
Starší služba\footnote{http://sciencesoft.at/latex/index?lang=en}, která mimo kompilace \LaTeX\ souborů vložených přes webový prohlížeč poskytuje i jiné rozhraní pro poskytnutí souborů, a to jmenovitě REST Api a SOAP. Jak bylo řečeno, tak tato služba je starší, tudíž pro kompilaci používá zastaralý TexLive 2008 a její vývoj není aktivní.

\subsection{ShareLaTeX}
Velmi podobný BlueLaTeXu, tedy open-source\footnote{https://github.com/sharelatex/clsi-sharelatex} nabízející jimi hostovanou verzi nebo lokální verzi pro vlastní potřebu, obě verze obsahují mnoho funkcí včetně grafického prostředí pro uživatele. Pod jménem ShareLaTeX se vyskytuje jenom výše zmíněné, ovšem společně s OverLeaf také spravují službu Pro\footnote{https://www.overleaf.com/for/enterprises}, která je určená pro firmy, ale i univerzity. Ta se pyšní možností nasazení na vlastních serverech, správou uživatelů, podporou a zabezpečením. 

\subsection{Porovnání}
Každé z těchto řešení má své problémy, ať už že je zastaralé a neaktualizované nebo nesplňující zanalyzované požadavky(sekce \ref{requirements}). Do první skupiny patří ScienceSoft a BlueLaTex, kde první z jmenovaných ani neposkytuje kód pro implementaci na lokálním serveru a druhý je v nedodělaném stavu, což může vést k nefunkčnosti a bezpečnostním rizikům. Služba Overleaf Commons nesplňuje stejný požadavek jako řešení od ScienceSoft, ale nabízí zajímavé prostředí pro vytváření a správu \LaTeX\ dokumentů pro studenty i zaměstnance, o čemž by univerzita mohla popřemýšlet. Nejnadějnější možností se zdá být ShareLaTeX, který poskytuje k použití i jenom backendovou část pro kompilaci \LaTeX\ a komunikaci, ale celý je implementovaný v CoffeeScriptu, což je v rozporu s požadavkem na implementaci, kde je Java EE. 

\section{Kompilátory}
Nejdůležitější částí celé aplikace bude kompilátor, který bude provádět kompilaci \LaTeX\ souborů, ale může poskytnout i podporu pro operace s výsledným PDF. Tyto operace vycházejí ze stanovených funkčních požadavků (sekce \ref{functional}). Jelikož kompilace bude probíhat na serveru s Linuxovým operačním systémem, nebudeme zahrnovat do srovnání kompilátory pro Windows.

\subsection{MikTeX}
MikTeX \footnote{https://miktex.org} je velmi oblíbený hlavně na operačním systému Windows, a to díky příjemné instalaci a kvůli možnosti doinstalovávat balíčky \enquote{on-the-fly}, neboli za běhu. Ale je poskytován mimo jiné  i pro Linux, a to například pro potřeby této práce v zajímavé verzi \enquote{Just enough TeX}. Tato instalace obsahuje jen to nejnutnější, tedy bez zbytečných balíčků navíc. Je vyvíjen jediným programátorem, který se stará o celou distribuci, což je do budoucna poněkud rizikové z pohledu udržitelnosti. Podporované Linuxové operační systémy jsou např. nejnovější Ubuntu a Debian.

\subsection{TeXLive}
TeXLive je kompilátor spravovaný skupinou přispěvatelů, kteří ho udržují. Existují dvě různé cesty jak TeXLive\footnote{https://www.tug.org/texlive/pkginstall.html} nainstalovat a od toho se odvíjí správa balíčků. Verze Native TeX Live a distribuce přímo pro daný operační systém. Liší se počtem balíčků po instalaci a způsobem jejich aktualizací. V Native verzi probíhají aktualizace pomocí TeX Live Manager neboli tlmgr, jinak se o to stará samotný operační systém. V případě verze Native je možno si i zvolit schéma, které určuje množství balíčků např. scheme-basic obsahuje jenom to nejnutnější. Podporuje skoro všechny Linuxové distribuce.

\subsection{Porovnání} \label{compilator}
Oba kompilátory jsou si velmi podobné a liší se jenom v drobnostech, některé z nich už byly nastíněny v jejich popisech. Jedním z důležitých aspektů je velikost instalace, tedy i s balíčky, v této kategorii nabízejí oba to samé. K tomuto se váže práce s balíčky, kde už je jiná situace, a to kvůli jasné výhodě MikTeXu spočívající  ve stahování balíčků za běhu. Něco podobného lze dosáhnout i v TexLive, ale je nutno použít externí skripty, což může mnohem více ovlivňovat rychlost kompilace. 
Rozdíl je také ve správě samotných kompilátorů, kde MikTeX je náchylnější k výpadkům aktualizací nebo celkovému zastavení vývoje, a to kvůli jedinému člověku, který se o něj stará. To může vést i k dalším problémům, co se týče bezpečnosti a podpory.
