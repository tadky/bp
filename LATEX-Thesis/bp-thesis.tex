\documentclass{ctuthesis}
%\usepackage{float}
\usepackage{csquotes}
\usepackage{multirow}
\usepackage{booktabs}
\usepackage{tabularx}
\newcommand{\tabitem}{~~\llap{\,\begin{picture}(-1,1)(-1,-3)\circle*{3}\end{picture}\ }~~}
\usepackage{array}
\usepackage{ragged2e}
\newcolumntype{P}[1]{>{\RaggedRight\hspace{0pt}}p{#1}}

\ctusetup{
	doctype-czech = {Bakalářská práce},
	xfaculty = F3,
	mainlanguage = czech,
	titlelanguage = czech,
	title-english = {Create REST service for conversion from LaTeX to PDF},
	title-czech = {Vytvoření REST služby pro konverzi LaTeX souborů do PDF},
	department-czech = {Katedra počítačů},
	author = {TADEÁŠ KYRAL},
	supervisor = {Ing. LUKÁŠ ZOUBEK},
	supervisor-address = {Katedra ekonomiky, manažerství a humanitních věd},
	fieldofstudy-czech = {Softwarové inženýrství a technologie},
	subfieldofstudy-czech = {žádné},
	keywords-czech = {PDF, LaTeX, PDF, Moodle, CourseWare, Web service, REST, Java EE},
	keywords-english = {PDF, LaTeX, PDF, Moodle, CourseWare, Web service, REST, Java EE},
	%front-specification = true,
	%specification-file = {}, % PDF s tvym zadanim prace
	day = 2,
	month = 11,
	year = 2018,
}

\ctuprocess

\addto\ctucaptionsczech{%
	\def\supervisorname{Vedoucí}%
}

\setcounter{tocdepth}{1}

% Podekovani
\begin{thanks}
	V první řadě bych chtěl poděkovat svému vedoucímu JMENO za příjemnou spolupráci a cenné rady při naších konzultacích. Dále bych chtěl poděkovat mé rodině a kamarádům, kteří mě
	při psaní této práce podporovali.
\end{thanks}

% Prohlaseni
\begin{declaration}
	Prohlašuji, že jsem předloženou práci vypracoval samostatně, a že jsem uvedl veškerou použitou literaturu.
	
	V Praze, \ctufield{day}.~\monthinlanguage{title}~\ctufield{year}
\end{declaration}


% Abstrakt v anglictine
\begin{abstract-english}
	Text follows...
\end{abstract-english}


% Abstrakt v cestine
\begin{abstract-czech}
	Práce pojednává o návrhu a vývoji webové služby pro konverzi LaTeXu do PDF, pro uživatele v Moodle a CourseWare. Doplněk má uživatelům usnadnit práci s LaTeXovými soubory přímo skrze portál Moodle, bez potřeby stahování a kompilace na svém stroji. Byla naimplementována služba na server a následně byla zapojena do provozu. Uživatelům portálů se usnadnila práce s LaTeX soubory. 
\end{abstract-czech}


\begin{document}
	\maketitle

	\chapter{Úvod}
LaTeX patří v akademickém prostředí mezi velmi oblíbené typografické systémy a je hojně využíván ke psaní skript a odborných prací. Mnoho akademiků využívá LaTeX i k vytváření materiálů pro studenty, primárně v oblasti matematiky, jelikož nabízí jednoduché nástroje ke psaní vzorců. Hlavním zdrojem materiálů pro studenty jsou portály Moodle a CourseWare, kam vyučující dokumenty nahrávají a můžou je tam i upravovat. Nynější editor podporuje jenom řádkové příkazy a není schopen zpracovat celý LaTeX dokument natož s více zdrojovými soubory. Tento stav mnoha učitelům nevyhovuje, protože by chtěli svoje dokumenty upravovat a přímo kompilovat v prohlížeči bez potřeby je stahovat a následně zase nahrávat. 

Práce se dělí na čtyři hlavní části. Nejdříve proběhne seznámení s potřebnou teorií a technologiemi, dále budou specifikovány požadavky na službu z čehož se najdou již podobné existující řešení a porovnají se s požadavky. Následně se přistoupí k návrhu samotné aplikace.


\section{Cíle práce}
Hlavním cílem bakalářské práce je implementovat webovou službu, která bude schopna komunikovat pomocí REST Api. Služba bude dělat konverzi LaTeX souborů do PDF. K tomu vedoucí jednotlivé dílčí cíle jako analýza existujících řešení, návrh, implementace a v neposlední řadě testování. 

	   
  \chapter{Teorie a technologie}
  V této části si přiblížíme technologie potřebné k návrhu a vývoji webové služby specifikované v zadání práce. Postupně budou vysvětleny všechny zásadní pojmy, které pomůžou čtenáři doplnit znalosti v dané problematice.
  
  \section{REST}
	Representational state transfer (dále jen REST) je architektura pro komunikaci  mezi distribuovanými systémy. Termín zavedl R. T. Fielding ve své disertační práci\cite{restThesis}, kde toto rozhraní bylo taktéž popsáno a vymezeno. Mimo jiné stanovil i těchto 5 základních pravidel, které by měly být dodrženy, aby se aplikace nazývala RESTful\cite{rest}:
	\begin{itemize}
		\item Klient-Server(Client-Server) - Toto omezení staví na principu oddělení zodpovědností (Separation of Concerns), nebo-li je klientská část starající se o uživatelské rozhraní a serverová část přistupující k databázi. Zlepšuje škálovatelnost sytému a zjednodušuje použitelnost na různých platformách.
		\item Bezestavovost(Stateless) - Každý požadavek musí přenášet všechna související data, server totiž neuchovává žádné informace o nynějším spojením a každý požadavek bere jako nový.
		\item Keš(Cache) - Data přenášená v odpovědi mohou být označená jako kešovatelná, tudíž si je klient může uložit a kdykoliv použít znova.
		\item Jednotné rozhraní(Unified interface) - Základem tohoto omezení je princip HATEOAS (Hypermedia As The Engine Of Application State), které říká, že klient nepotřebuje znát pravidla komunikace dopředu a data musí obsahovat odkazy na další data v aplikaci. Měla by být jasně definovaná adresa zdroje (např. URI), reprezentace přenášených dat (např. HTML), typ média (např. JSON)
		\item Vrstvený systém(Layered system) - Přidáním vrstev se aplikace, kde každá vrstva je izolovaná a může komunikovat jenom se sousedícími vrstvami, zpřehlední a zlepší se její škálovatelnost.
	\end{itemize}	
	Nejčastějším typem protokolu využívající tuto architekturu je Hypertext Transfer Protocol(HTTP). Pomocí čtyř hlavních metod GET, PUT, POST, DELETE v požadavku poslaného z klienta server buď odpovídajíce vrátí požadovaná data, přijme data poslané v těle a uloží do databáze, nebo data vymaže. 

\section{PDF}
	Portable Document Format(dále jen PDF), jak už název napovídá, jedná se o formát dokumentů, jejichž hlavním cílem je poskytování nazávislosti na platformě. Velmi podobný programovacímu jazyku PostScript z kterého vzešel. Syntaxe souboru je nejlépe pochopitelná jako čtyři komponenty: 
	\begin{itemize}
		\item Objekty(Objects) - Základní jednotka celého souboru např. Pole, Čísla, Řetězce znaků. Jednotlivé objekty se popisují množinou znaků, která je definována lexikálními konvencemi.
		\item Struktura souboru(File Structure) - Samotný soubor se skládá ze čtyř částí, jak je i vidět na obrázku \ref{fig:pdf}, ty si lehce popíšeme: 
			\begin{itemize}
				\item hlavička - identifikuje verzi PDF
				\item tělo - obsahuje použité objekty, které reprezentují obsah dokumentu
				\item tabulka odkazů - obsahuje odkazy na objekty v podobě počtu bytů od začátku souboru, kvůli náhodnému přístupu bez potřeby číst celý soubor
				\item závěrečná sekce - udává pozici tabulky odkazů a speciálních objektů
			\end{itemize}
		Tato struktura napomáhá k náhodnému přístupu k jednotlivým částem a usnadňuje jejich aktualizaci.
		\item Struktura dokumentu(Document Structure) - Popisuje hierarchickou strukturu objektů v těle dokumentu.
		\item Content stream - Objekt v kterém se nacházejí instrukce k vykreslování grafických elementů. Každá stránka má minimálně jeden a na rozdíl od ostatních objektu je procházen sekvenčně.	
	\end{itemize}

	\newpage
	\begin{figure}[H]
		\includegraphics[scale=0.9]{Untitled}
		\centering
		\caption{Struktura objektů}
		\label{fig:pdf}
	\end{figure}

 \section{LaTeX} 
	Vychází z typografického sázecího systému TeX, který popisuje Pavel Satrapa ve své knize \cite{latex}: \enquote{\textit{Patří do rodiny tak zvaných značkovacích jazyků (markup languages) a dal by se zjednodušeně charakterizovat jako programovací jazyk pro sazbu textů. Jeho základním vstupem je textový soubor, který obsahuje jak sázený dokument, tak příkazy ovlivňující sazbu. Určité znaky mají přiřazen speciální význam a jejich prostřednictvím jsou v textu odlišeny řídicí konstrukce. Typickým příkladem je zpětné lomítko, jímž začínají příkazy.}}
	
	LaTex je rozšíření TeXu o balíček přednastavených řídících konstrukcí. Hlavní cílem těchto systému je jednoduchost pro psaní matematických a jiných vzorců. Ovšem je také velmi oblíben kvůli silné možnosti jednoduše upravovat dokumenty ke svému zalíbení, přestože prvotní seznámení je náročnější oproti jiným nástrojům ke psaní.
	
	Soubor se skládá z preambule, která obsahuje nastavení pro celý dokument, toto nastavení může být v jiném souboru a případně používané balíčky. Druhá část je dokumentu v kterém se mimo jiné nachází kapitoly, sekce a podobně. Základní syntaxe může vypadat takto.
	\begin{verbatim}
	\documentclass{thesis}
	\usepackage{csquotes}
	
	\begin{document}
	Hello world!
	\end{document}
	\end{verbatim}
	
	Základní soubory mají příponu tex kromě souboru s nastavením, ten je ve formátu cls, neboli class file. Tyto soubory mohou být uspořádány do stromové struktury, kde v kořenu stromu je jeden hlavní soubor, do kterého jsou vnořovány další, ale vše může být i v jednom. To napomáhá přehlednosti velkých dokumentů a také používání již vytvořených. 
	
	Pro získání výsledného dokumentu, například ve formátu PDF, je potřeba vše zkompilovat. To provádí kompilátor podle nastavení.  

\section{FURPS+}
	Vychází z klasifikace požadavků FURPS(\textbf{F}unctionality, \textbf{U}sability, \textbf{R}eliability, \textbf{P}erformance, \textbf{S}upportability) s kterou přišel Robert Grady v roce 1992. Roku 1999, Jacobson at el rozšířili specifikaci o znaménko "+", které přidává požadavky a omezení na návrh, implementaci, rozhraní a hardware. 
	
	\begin{itemize}
		\item Funkční požadavky
			\begin{itemize}
				\item Fukčnost(Functionality) - Požadavky popisující všechny hlavní prvky produktu i důležité aspekty z pohledu architektury např. lokalizovaný systém pro více jazyků.
			\end{itemize}
		\item Nefunknční požadavky
			\begin{itemize}
				\item Použitelnost(Usability) - Zaměřuje se na uživatelskou přívětivost nejenom samotné aplikace, ale i dokumentace, týkající se estetiky a konzistence.
				\item Spolehlivost(Reliability) - Spolehlivost systému v podobě doby běhu, správnosti fungování a četnosti výpadků.
				\item Výkon(Performance) - Vypovídá o výkonnosti systému, jak rychle dokáže zpracovávat požadavky, spustit se atd.
				\item Podporovatelnost(Supportability) - Popisuje testovatelnost, škálovatelnost, konfigurovatelnost...
				\item Návrh(Design) - Omezení na návrh systému např. požadavek na relační databázi
				\item Implementace(Implementation) - Specifikuje typ programovacího jazyku, platformu apod.
				\item Rozhraní(Interface) - Komunikace s externími systémy
				\item Fyzické(Physical) - Definuje požadavky na hardware, na kterém daný software poběží i co se týče fyzické velikosti
			\end{itemize}
	\end{itemize}
	
	Toto rozdělení nám pomáhá identifikovat požadavky. Přispívá k vyšší kvalitě systému a snižuje pravděpodobnost přehlédnutí funkcionality. Právě díky těmto vlastnostem je velmi oblíbené a využívané k vývoji jakéhokoliv software. 
 
	 \chapter{Analýza}
 
 
 \section{Požadavky}
 Nyní představíme požadavky kladené na aplikaci. Ty jsou buď požadované zadávajícím nebo odvozené z potřeb, ke kterým bude služba používána.Ale i z omezení plynoucích z prostředí v jakém bude provozována, v tomto případě pro fakultu státní vysoké školy, jmenovitě Fakulta elektrotechnická, ČVUT. \\
 Za pomoci metody FURPS+ rozdělíme požadavky a omezení na funkční a nefunkční. 
 
 \subsection{Funkční požadavky}
 \begin{table}[H]
 	\begin{center}
 		\begin{tabular}{ p{4cm} P{0,7cm} P{7,9cm} }
 			\textbf{Typ} & \multicolumn{2}{l}{\textbf{Požadavky}} \\
 			\midrule[0,15em]
 			Funkčnost 
 			&\multicolumn{2}{l}{\tabitem Webová služba s REST rozhraním pro komunikaci} \\
 			&\multicolumn{2}{l}{\tabitem Umět přijmout požadavky a soubory}\\
 			&\multicolumn{2}{P{7,6cm}}{\tabitem Zkompilování LaTeX souborů s požadovaným nastavením do PDF}\\
 			&\multicolumn{2}{l}{\tabitem Uložení PDF a jeho poskytnutí}\\	
 			&\multicolumn{2}{l}{\tabitem Uživatel může zvolit tyto nastavení:}\\
 			&&\tabitem Přidat vodoznak na kařdou stranu PDF\\
 			&&\tabitem Nakonec souboru přidat stránku s předem stanoveným obsahem\\
 			&&\tabitem Výsledný soubor PDF bude chráněný\\
 			&&\tabitem Výsledný soubor PDF nebude kopírovatelný nebo tisknutelný\\
 		\end{tabular}
 	\end{center}
 	\caption{Funkční požadavky}
 	\label{tab:errors}
 \end{table} 
 Služba ma jasný účel a z toho i plyne menší množství funkcionalit.  
 
 \subsection{Nefunkční požadavky}
 
 \begin{table}[H]
 	\begin{center}
 		\begin{tabular}{ p{4cm} P{8,6cm} }
 			\textbf{Typ} & \textbf{Požadavky} \\
 			\midrule[0,15em]
 			Použitelnost & \tabitem Dokumentace k rozhraní na swagger.com\\
 						& \tabitem Přístup skrz REST api, pouze s tokenem\\
 			\midrule		
 			Spolehlivost & \tabitem Služba není kritická\\
 										& \tabitem Autorizace pomocí OAuth\\
 			\midrule
 			Výkon & \tabitem Zvládnutí obsloužení desítky požadavků najednou\\
 						& \tabitem Ukládání výsledných dokumentů po dobu jednoho měsíce\\
 						& \tabitem Maximální doba tvorby dokumentu 5 minut\\	
			\midrule
			Podporovatelnost & \tabitem Dokumentace k rozhraní na swagger.com\\
							& \tabitem Služba může být rozšířena o kompilování samotného TeXu a může podporovat vytváření i jiných formátů z LaTeXu\\	
			\midrule
			Implementace & \tabitem Platforma - Java EE\\
						& \tabitem Komunikace - REST Api\\
			\midrule
 			Rozhraní & \tabitem Komunikuje s Moodle a CourseWare, tyto portály posílají data\\
 			\midrule
 			Fyzické & \tabitem Musí být provozováno na serverech ČVUT
 	\end{tabular}
 	\end{center}
 	\caption{Funkční a nefunkční požadavky}
 	\label{tab:errors}
 \end{table}
 
 Z tabulky je vidět, že služba není nijak kritická a působí jenom jako doplněk do výše zmíněných portálů. Musí ovšem splňovat vyšší bezpečnostní nároky zapříčiněné prostředím v jakém se bude používat. 

\section{Existující řešení}
Na základě stanovených požadavků v minulé kapitole přejdeme k analýze již existujících řešení. Mometálně uživatelé Moodle mohou vkládat do svých souborů řádkové příkazy, což není úplně dostačující a nesplňuje to požadavky. Samozřejmě se dají používat pro vytváření PDF z LaTeX souborů kompilátory, které si můžete stáhnout a používat lokálně. To ovšem není předmětem této práce, a proto se podíváme na webové služby, které více odpovídají potřebám. Pár vybraných si představíme.  

\subsection{OverLeaf}
https://www.overleaf.com/for/universities
Placená služba pro tvorbu LaTeX dokumentů, která umožňuje i používání s omezeními zadarmo. Je velmi oblíbená hlavně kvůli hezkému prostředí pro tvorbu dokumentů a jejich správu. Také nabízí výhody pro určité zájmové skupiny, nejzajimavější vzhledem k tématu této práce je předplatitelská služba OverLeaf Commons https://www.overleaf.com/for/universities. Ta poskytuje sdílené prostředí se všemi výhodami pro zaměstnance a studenty univerzity.

\subsection{BlueLaTeX}
http://www.bluelatex.org/
Open source služba pro kompilaci LateX souborů. Kompilovat můžete na jejich serveru nebo nabízejí kód pro spuštění na vlastním. Kromě samotné serverové implementace je k dispozici i webový klient. Vše je pouze zatím v Beta verzi a samotní tvůrci varují před možnými nedostatky a problémy. Hlavním lákadlem je real time spolupráce více lidí na jednom dokumentu a také možnost provozovat server lokálně. 

\subsection{ScienceSoft}
http://sciencesoft.at/latex/index?lang=en
Starší služba, která mimo kompilace LaTex souborů vložených přímo na stránky poskytuje i jiné rozhraní pro posílání souborů a to skrz REST Api a SOAP. Jak bylo řečeno, tak tato služba je už starší tudíž pro kompilaci používá TexLive 2008.

\subsection{ShareLaTeX}\\
https://github.com/sharelatex/clsi-sharelatex
Velmi podobný BlueLaTeX, tedy open source nabízející jejich hostovanou verzi nebo lokální verzi pro vlastní potřebu, obě verze obsahují mnoho funkcí včetně grafického prostředí pro uživatele. Pod jménem ShareLaTeX se vyskytuje jenom výše zmíněné, https://www.overleaf.com/for/enterprises ovšem společně s OverLeaf také spravují službu Pro, která je určená pro firmy, ale i univerzity. Ta se pyšní možností nasazení na vlastních serverech, správou uživatelů, podporou a zabezpečením.

\subsection{Porovnání}
Každé z těchto řešení ma své problémy, ať už že je zastaralé a neaktualizované nebo nesplňují zanalyzované požadavky(kapitola 3.1). Do první skupiny patří ScienceSoft a BlueLaTex, kde první z jmenovaných ani neposkytuje kód pro implementaci na lokálním serveru a druhý je v nedodělaném stavu, což může vést k nefunkčnosti a bezpečnostním rizikům. Služba Overleaf Commons nesplňuje stejný požadavek jako řešení od ScienceSoft, ale nabízí zajímavé prostředí pro vytváření a správu LaTeX dokumentů pro studenty i zaměstnance, o čemž by univerzita mohla popřemýšlet. Nejnadějnější možností se zdá být ShareLaTeX, který poskytuje k použití i jenom backend pro kompilaci LaTeX a komunikaci, ale celý je implementovaný v CoffeeScriptu, což si rozporuje s požadavkem na implementaci, kde je Java EE. 

\section{Kompilátory}
Nejdůležitější částí celé aplikace bude kompilátor, který samozřejmě provádí kompilaci, ale může poskytnout i podporu pro operace s výsledným PDF. Tyto operace vycházejí ze stanovených funkčních požadavků(kapitola 3.1.1). Jelikož kompilace bude probíhat na serveru, kde bude nainstalován Linuxový operační systém výběr se zužuje na jedinou možnost, kterou je TeXLive.

\subsection{TeXLive}
https://www.tug.org/texlive/pkginstall.html
Existují dvě různé cesty jak nainstalovat TeXLive a od toho se odvíjí správa balíčků. Verze Native TeX Live a distribuce přímo pro daný operační systém. Liší se ve způsobu aktualizací balíčků a základní po instalaci. V Native verzi probíhají aktualizace pomocí TeX Live Manager nebo-li tlmgr 

	  
  	\chapter{Návrh}  

V této části popíšeme používané technologie

\section{Návrh}
Na základě požadavků bude web služba postavena na Java EE, což je platforma pro vývoj webových aplikací, rozšiřující standardní Javu SE. Představíme si alespoň jednu základní techniku, kterou poskytuje navíc. 
\par
\textbf{Vkládání závislostí} (dependency injection) umožňuje objektu používat jiné objekty bez potřeby ho zatěžovat jejich vytvářením. Obekty, které můžeme takto vkládat se nazývají beany a právě o jejich vytváření a zánik se stará Contexts and Dependency Injection(dále jen CDI) kontejner.

Dále je potřeba specifikovat aplikační server, ten poskytuje pro webové aplikace běhové prostředí, tedy zajišťuje správu databázových spojení apod. Na základě zkušeností je vybrán open-source Payara, který staví na GlassFish, o proti němu poskytuje častější aktualizace a opravy chyb. 



\section{Architektura}
  

	\begin{center}
		\textit{R = (A $\lor$ B $\lor$ C $\lor$ D)}
	\end{center}
 
 	\begin{figure}[H]
 		\includegraphics[width=0.9\textwidth]{process_app}
 		\caption{Vizualizace průchodu aplikací}
	 \end{figure}
	
	\chapter{Implementace}

	\chapter{Závěr}
	Text follows
	
	\bibliographystyle{acm}
	\bibliography{ctubib}
	
	\appendix

\chapter{Seznam zdrojů}

\medskip
\bgroup \leftskip=5.5em

\begin{itemize}
	
	\item Obrázek \ref{fig:pdf} -- Převzato z dokumentace PDF\cite{pdf} 25. 11. 2018.
	\item Obrázek \ref{fig:seq} -- Vytvořeno autorem 30. 12. 2018.
	
\end{itemize}

\par\egroup
	
	\appendix

\chapter{Seznam zkratek}

\medskip
\bgroup \leftskip=5.5em

\begin{description}

	\item[LMS] -- Learning Management System
	
	\item[SWOT] -- Strengths, Weaknesses, Opportunities, Threats
	
	\item[GDPR] -- General Data Protection Regulation
	
	\item[HTTP] -- Hypertext Transfer Protocol
	
	\item[API] -- Aplikační rozhraní
	
	\item[CW] -- CourseWare
	
	\item[JPA] -- Java Persistance API
	
	\item[JDBC] -- Java Database Connectivity	
	

\end{description}

\par\egroup
	
	\input{prilohy}
	
\end{document}