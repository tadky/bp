\chapter{Úvod}
LaTeX patří v akademickém prostředí mezi velmi oblíbené typografické systémy a je hojně využíván ke psaní skript a odborných prací. Mnoho akademiků využívá LaTeX i k vytváření materiálů pro studenty, primárně v oblasti matematiky, jelikož nabízí jednoduché nástroje ke psaní vzorců. Hlavním zdrojem materiálů pro studenty jsou portály Moodle a CourseWare, kam vyučující dokumenty nahrávají a můžou je tam i upravovat. Nynější editor podporuje jenom řádkové příkazy a není schopen zpracovat celý LaTeX dokument natož s více zdrojovými soubory. Tento stav mnoha učitelům nevyhovuje, protože by chtěli svoje dokumenty upravovat a přímo kompilovat v prohlížeči bez potřeby je stahovat a následně zase nahrávat. 

Práce se dělí na čtyři hlavní části. Nejdříve proběhne seznámení s potřebnou teorií a technologiemi, dále budou specifikovány požadavky na službu z čehož se najdou již podobné existující řešení a porovnají se s požadavky. Následně se přistoupí k návrhu samotné aplikace.


\section{Cíle práce}
Hlavním cílem bakalářské práce je implementovat webovou službu, která bude schopna komunikovat pomocí REST Api. Služba bude dělat konverzi LaTeX souborů do PDF. K tomu vedoucí jednotlivé dílčí cíle jako analýza existujících řešení, návrh, implementace a v neposlední řadě testování. 