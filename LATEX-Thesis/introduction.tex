\chapter{Úvod}
\LaTeX\ patří v akademickém prostředí mezi velmi oblíbené typografické systémy a je hojně využíván ke psaní skript a odborných prací. Mnoho akademiků využívá \LaTeX\ i k vytváření materiálů pro studenty, primárně v oblasti matematiky, jelikož nabízí jednoduché nástroje ke psaní vzorců. Hlavním zdrojem materiálů pro studenty jsou portály Moodle a CourseWare, kam vyučující dokumenty nahrávají a můžou je tam i upravovat. Nynější editor podporuje jenom řádkové příkazy a není schopen zpracovat celý \LaTeX\ dokument natož s více zdrojovými soubory. Tento stav mnoha učitelům nevyhovuje, protože by chtěli svoje dokumenty upravovat a přímo kompilovat v prohlížeči bez potřeby je stahovat a následně zase nahrávat. 

Práce se dělí na několik částí, které je potřeba splnit k úspěšnému dokončení projektu. Nejdříve proběhne seznámení s potřebnou teorií a technologiemi, dále budou specifikovány požadavky na službu, poté budou představena již podobná existující řešení a porovnají se s požadavky. Následně se přistoupí k návrhu samotné aplikace na základě něhož bude naimplementována. Nakonec bude služba otestována a zhodnocena.


\section{Cíle práce}
Hlavním cílem práce je poskytnout konverzi \LaTeX\ souborů do PDF pro uživatele Moodle a CourseWare. V semestrálním projektu budou rozpracovány tyto dva dílčí cíle: analýza a návrh.

\subsection{Analýza} 
V analýze jsou stanoveny tyto cíle:

\begin{itemize}
	\item Získat a zformulovat požadavky na službu.
	\item Naleznout již eistující řešení, zabývající se touto problematikou a porovnat je vůči požadavkům.
	\item Porovnat \LaTeX\ kompilátory. 
\end{itemize}
\newpage
\subsection{Návrh}
V návrhu budou naplněny tyto cíle:

\begin{itemize}
	\item Navrhnout řešení daného problému. 
	\item Na základě požadavků stanovit technologie.
\end{itemize}

\subsection{Implementace}
V rámci implementace budou dosaženy tyto cíle:

\begin{itemize}
	\item Naimplementovat funkční aplikaci
	\item Otestovat aplikaci
\end{itemize}



